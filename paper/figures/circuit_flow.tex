% figures/circuit_flow.tex -- Illustrative circuit-level data flow (TikZ)
% Shows residual stream layers with SAE features and ACD-discovered circuit path

\begin{tikzpicture}[
    font=\footnotesize,
    layer/.style = {
        rectangle, draw=black!50, fill=gray!6,
        minimum width=1.2cm, minimum height=3.8cm,
        rounded corners=2pt, line width=0.5pt
    },
    feat/.style = {
        circle, draw=black!40, fill=white,
        minimum size=0.42cm, inner sep=0pt, line width=0.4pt
    },
    highfeat/.style = {
        circle, draw=red!70!black, fill=red!18,
        minimum size=0.42cm, inner sep=0pt, line width=0.7pt
    },
    circfeat/.style = {
        circle, draw=blue!70!black, fill=blue!20,
        minimum size=0.48cm, inner sep=0pt, line width=1pt
    },
    resid/.style = {
        -{Latex[length=1.6mm]}, thick, draw=black!50
    },
    circedge/.style = {
        -{Latex[length=1.8mm]}, very thick, draw=blue!70!black
    },
    layerlabel/.style = {font=\scriptsize, text=black!60}
]

\def\layersep{1.7}

% Draw 6 residual stream layers (5-10)
\foreach \li [count=\xi] in {5,6,7,8,9,10} {
    \pgfmathsetmacro\xpos{(\xi-1)*\layersep}
    \node[layer] (L\li) at (\xpos, 0) {};
    \node[layerlabel, below=0.05cm of L\li] {$l=\li$};
    % Residual stream arrow
    \ifnum\xi<6
        \pgfmathsetmacro\xnext{\xi*\layersep}
        \draw[resid] (\xpos+0.6, 0.1) -- (\xnext-0.6, 0.1);
    \fi
}

% Add inactive features to each layer (grey dots)
\foreach \li [count=\xi] in {5,6,7,8,9,10} {
    \pgfmathsetmacro\xpos{(\xi-1)*\layersep}
    \foreach \yi in {-1.4,-0.9,-0.4,0.1,0.6,1.1,1.6} {
        \node[feat] at (\xpos, \yi) {};
    }
}

% Highlight active features (orange = active, not in circuit)
\node[highfeat] (f5a) at (0*\layersep, 1.1) {};
\node[highfeat] (f6a) at (1*\layersep, 0.6) {};
\node[highfeat] (f6b) at (1*\layersep, -0.4) {};
\node[highfeat] (f7a) at (2*\layersep, 1.1) {};
\node[highfeat] (f8a) at (3*\layersep, 0.1) {};
\node[highfeat] (f9a) at (4*\layersep, -0.9) {};

% Highlight circuit members (blue = selected by ACD)
\node[circfeat] (c5)  at (0*\layersep, -0.4) {};
\node[circfeat] (c6)  at (1*\layersep, 1.6) {};
\node[circfeat] (c7)  at (2*\layersep, -0.9) {};
\node[circfeat] (c8)  at (3*\layersep, 1.1) {};
\node[circfeat] (c9)  at (4*\layersep, 0.6) {};
\node[circfeat] (c10) at (5*\layersep, -0.4) {};

% Circuit edges (causal pathway)
\draw[circedge] (c5)  -- (c6);
\draw[circedge] (c6)  -- (c7);
\draw[circedge] (c7)  -- (c8);
\draw[circedge] (c8)  -- (c9);
\draw[circedge] (c9)  -- (c10);

% Legend
\node[feat,  label={right:\scriptsize Active, non-circuit}]
    at ($(L10.east)+(0.6, 0.8)$) {};
\node[highfeat, label={right:\scriptsize Active, high EFE}]
    at ($(L10.east)+(0.6, 0.2)$) {};
\node[circfeat, label={right:\scriptsize Circuit member (\ACD)}]
    at ($(L10.east)+(0.6,-0.4)$) {};

% Input label
\node[font=\scriptsize, align=center] at (-1.0, 0.1) {``The\\Golden\\Gate..''};
\draw[-{Latex[length=1.5mm]}, black!50] (-0.5, 0.1) -- (L5.west |- 0,0.1);

% Output label
\node[font=\scriptsize, align=center] at (5*\layersep+1.0, 0.1) {``San''\\(target)};
\draw[-{Latex[length=1.5mm]}, black!50] (L10.east |- 0,0.1) -- (5*\layersep+0.6, 0.1);

\end{tikzpicture}
