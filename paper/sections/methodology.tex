This section formalises the \ACD{} framework, detailing the attribution
graph backend, the active inference-inspired feature selector, and the
intervention engine.

\subsection{Architecture Overview}

\begin{figure}[t]
\centering
\resizebox{\columnwidth}{!}{%
% figures/architecture.tex -- System architecture diagram (TikZ)
% Rendered as % figures/architecture.tex -- System architecture diagram (TikZ)
% Rendered as % figures/architecture.tex -- System architecture diagram (TikZ)
% Rendered as \input{figures/architecture} inside a figure environment

\begin{tikzpicture}[
    font=\small,
    node distance = 0.9cm and 1.5cm,
    box/.style = {
        rectangle, draw=black!70, fill=blue!8,
        rounded corners=3pt, minimum width=2.6cm, minimum height=0.7cm,
        align=center, line width=0.5pt
    },
    redbox/.style = {
        rectangle, draw=red!60!black, fill=red!8,
        rounded corners=3pt, minimum width=2.6cm, minimum height=0.7cm,
        align=center, line width=0.5pt
    },
    greenbox/.style = {
        rectangle, draw=green!50!black, fill=green!8,
        rounded corners=3pt, minimum width=2.6cm, minimum height=0.7cm,
        align=center, line width=0.5pt
    },
    greybox/.style = {
        rectangle, draw=black!40, fill=black!4,
        rounded corners=3pt, minimum width=2.6cm, minimum height=0.7cm,
        align=center, line width=0.5pt
    },
    arr/.style = {-{Latex[length=2mm]}, thick, draw=black!60},
    darr/.style = {{Latex[length=2mm]}-{Latex[length=2mm]}, thick, draw=black!40, dashed},
    groupbox/.style = {
        draw=black!30, dashed, fill=none, rounded corners=6pt, inner sep=6pt
    }
]

% ---- Input ----
\node[box] (prompt) {Prompt $x$};

% ---- LLM column ----
\node[greybox, below=0.7cm of prompt] (tlens) {TransformerLens\\\texttt{HookedTransformer}};
\node[greybox, below=0.6cm of tlens] (cache) {Residual stream\\cache $\{\vect{h}_l\}$};
\node[greybox, below=0.6cm of cache] (saelens) {SAE-Lens\\\texttt{SAE.from\_pretrained}};
\node[greybox, below=0.6cm of saelens] (features) {Active features\\$\mathcal{F}(x)$};

% ---- AI Agent column (right) ----
\node[redbox, right=2.4cm of tlens] (beliefs) {Belief state\\$q(\vect{s}_t)$};
\node[redbox, below=0.6cm of beliefs] (efe) {\EFE\ selection\\Eq.~(\ref*{eq:efe_approx})};
\node[redbox, below=0.6cm of efe] (action) {Intervention\\$(l^*,k^*,u^*)$};
\node[redbox, below=0.6cm of action] (observe) {Observe\\$o_t$, $\delta_t$};
\node[redbox, below=0.6cm of observe] (update) {Belief update\\Eq.~(\ref*{eq:belief_update})};

% ---- Output column ----
\node[greenbox, below=0.7cm of features, xshift=3.0cm] (circuit) {Circuit $\mathcal{C}$};
\node[greenbox, below=0.5cm of circuit] (graph) {Attribution\\graph $G$};
\node[greenbox, below=0.5cm of graph] (preds) {Predictions $\mathcal{P}$};

% ---- pymdp label ----
\node[above=0.1cm of beliefs, font=\footnotesize\itshape, text=red!70!black]
    {\pymdp\ Agent};

% ---- Grouping boxes ----
\begin{scope}[on background layer]
  \node[groupbox, fit=(tlens)(cache)(saelens)(features),
        label={[font=\scriptsize, text=black!50]above:LLM + SAE pipeline}] {};
  \node[groupbox, draw=red!40, fit=(beliefs)(efe)(action)(observe)(update),
        label={[font=\scriptsize, text=red!60!black]above:Active Inference loop}] {};
  \node[groupbox, draw=green!40, fit=(circuit)(graph)(preds),
        label={[font=\scriptsize, text=green!60!black]above:Outputs}] {};
\end{scope}

% ---- Arrows ----
\draw[arr] (prompt) -- (tlens);
\draw[arr] (tlens) -- (cache);
\draw[arr] (cache) -- (saelens);
\draw[arr] (saelens) -- (features);

% features -> beliefs (initialise)
\draw[arr] (features.east) -- ++(0.4,0) |- (beliefs.west)
    node[midway, above, font=\scriptsize]{init};

% EFE loop
\draw[arr] (beliefs) -- (efe);
\draw[arr] (efe) -- (action);
\draw[arr] (action.west) -- ++(-0.5,0) |- (cache.east)
    node[pos=0.3, right, font=\scriptsize]{intervene};
\draw[arr] (cache.east) -- ++(0.5,0) |- (observe.west)
    node[pos=0.5, right, font=\scriptsize]{$o_t$};
\draw[arr] (observe) -- (update);
\draw[arr] (update.north) -- ++(0, 0.15) -| (beliefs.south)
    node[pos=0.1, right, font=\scriptsize]{$q_{t+1}$};

% Convergence check
\draw[darr] (update.east) -- ++(0.5,0)
    node[right, font=\scriptsize]{converged?};

% Output arrows
\draw[arr] (update.south) -- ++(0,-0.3) -| (circuit.north);
\draw[arr] (circuit) -- (graph);
\draw[arr] (graph) -- (preds);

\end{tikzpicture}
 inside a figure environment

\begin{tikzpicture}[
    font=\small,
    node distance = 0.9cm and 1.5cm,
    box/.style = {
        rectangle, draw=black!70, fill=blue!8,
        rounded corners=3pt, minimum width=2.6cm, minimum height=0.7cm,
        align=center, line width=0.5pt
    },
    redbox/.style = {
        rectangle, draw=red!60!black, fill=red!8,
        rounded corners=3pt, minimum width=2.6cm, minimum height=0.7cm,
        align=center, line width=0.5pt
    },
    greenbox/.style = {
        rectangle, draw=green!50!black, fill=green!8,
        rounded corners=3pt, minimum width=2.6cm, minimum height=0.7cm,
        align=center, line width=0.5pt
    },
    greybox/.style = {
        rectangle, draw=black!40, fill=black!4,
        rounded corners=3pt, minimum width=2.6cm, minimum height=0.7cm,
        align=center, line width=0.5pt
    },
    arr/.style = {-{Latex[length=2mm]}, thick, draw=black!60},
    darr/.style = {{Latex[length=2mm]}-{Latex[length=2mm]}, thick, draw=black!40, dashed},
    groupbox/.style = {
        draw=black!30, dashed, fill=none, rounded corners=6pt, inner sep=6pt
    }
]

% ---- Input ----
\node[box] (prompt) {Prompt $x$};

% ---- LLM column ----
\node[greybox, below=0.7cm of prompt] (tlens) {TransformerLens\\\texttt{HookedTransformer}};
\node[greybox, below=0.6cm of tlens] (cache) {Residual stream\\cache $\{\vect{h}_l\}$};
\node[greybox, below=0.6cm of cache] (saelens) {SAE-Lens\\\texttt{SAE.from\_pretrained}};
\node[greybox, below=0.6cm of saelens] (features) {Active features\\$\mathcal{F}(x)$};

% ---- AI Agent column (right) ----
\node[redbox, right=2.4cm of tlens] (beliefs) {Belief state\\$q(\vect{s}_t)$};
\node[redbox, below=0.6cm of beliefs] (efe) {\EFE\ selection\\Eq.~(\ref*{eq:efe_approx})};
\node[redbox, below=0.6cm of efe] (action) {Intervention\\$(l^*,k^*,u^*)$};
\node[redbox, below=0.6cm of action] (observe) {Observe\\$o_t$, $\delta_t$};
\node[redbox, below=0.6cm of observe] (update) {Belief update\\Eq.~(\ref*{eq:belief_update})};

% ---- Output column ----
\node[greenbox, below=0.7cm of features, xshift=3.0cm] (circuit) {Circuit $\mathcal{C}$};
\node[greenbox, below=0.5cm of circuit] (graph) {Attribution\\graph $G$};
\node[greenbox, below=0.5cm of graph] (preds) {Predictions $\mathcal{P}$};

% ---- pymdp label ----
\node[above=0.1cm of beliefs, font=\footnotesize\itshape, text=red!70!black]
    {\pymdp\ Agent};

% ---- Grouping boxes ----
\begin{scope}[on background layer]
  \node[groupbox, fit=(tlens)(cache)(saelens)(features),
        label={[font=\scriptsize, text=black!50]above:LLM + SAE pipeline}] {};
  \node[groupbox, draw=red!40, fit=(beliefs)(efe)(action)(observe)(update),
        label={[font=\scriptsize, text=red!60!black]above:Active Inference loop}] {};
  \node[groupbox, draw=green!40, fit=(circuit)(graph)(preds),
        label={[font=\scriptsize, text=green!60!black]above:Outputs}] {};
\end{scope}

% ---- Arrows ----
\draw[arr] (prompt) -- (tlens);
\draw[arr] (tlens) -- (cache);
\draw[arr] (cache) -- (saelens);
\draw[arr] (saelens) -- (features);

% features -> beliefs (initialise)
\draw[arr] (features.east) -- ++(0.4,0) |- (beliefs.west)
    node[midway, above, font=\scriptsize]{init};

% EFE loop
\draw[arr] (beliefs) -- (efe);
\draw[arr] (efe) -- (action);
\draw[arr] (action.west) -- ++(-0.5,0) |- (cache.east)
    node[pos=0.3, right, font=\scriptsize]{intervene};
\draw[arr] (cache.east) -- ++(0.5,0) |- (observe.west)
    node[pos=0.5, right, font=\scriptsize]{$o_t$};
\draw[arr] (observe) -- (update);
\draw[arr] (update.north) -- ++(0, 0.15) -| (beliefs.south)
    node[pos=0.1, right, font=\scriptsize]{$q_{t+1}$};

% Convergence check
\draw[darr] (update.east) -- ++(0.5,0)
    node[right, font=\scriptsize]{converged?};

% Output arrows
\draw[arr] (update.south) -- ++(0,-0.3) -| (circuit.north);
\draw[arr] (circuit) -- (graph);
\draw[arr] (graph) -- (preds);

\end{tikzpicture}
 inside a figure environment

\begin{tikzpicture}[
    font=\small,
    node distance = 0.9cm and 1.5cm,
    box/.style = {
        rectangle, draw=black!70, fill=blue!8,
        rounded corners=3pt, minimum width=2.6cm, minimum height=0.7cm,
        align=center, line width=0.5pt
    },
    redbox/.style = {
        rectangle, draw=red!60!black, fill=red!8,
        rounded corners=3pt, minimum width=2.6cm, minimum height=0.7cm,
        align=center, line width=0.5pt
    },
    greenbox/.style = {
        rectangle, draw=green!50!black, fill=green!8,
        rounded corners=3pt, minimum width=2.6cm, minimum height=0.7cm,
        align=center, line width=0.5pt
    },
    greybox/.style = {
        rectangle, draw=black!40, fill=black!4,
        rounded corners=3pt, minimum width=2.6cm, minimum height=0.7cm,
        align=center, line width=0.5pt
    },
    arr/.style = {-{Latex[length=2mm]}, thick, draw=black!60},
    darr/.style = {{Latex[length=2mm]}-{Latex[length=2mm]}, thick, draw=black!40, dashed},
    groupbox/.style = {
        draw=black!30, dashed, fill=none, rounded corners=6pt, inner sep=6pt
    }
]

% ---- Input ----
\node[box] (prompt) {Prompt $x$};

% ---- LLM column ----
\node[greybox, below=0.7cm of prompt] (tlens) {TransformerLens\\\texttt{HookedTransformer}};
\node[greybox, below=0.6cm of tlens] (cache) {Residual stream\\cache $\{\vect{h}_l\}$};
\node[greybox, below=0.6cm of cache] (saelens) {SAE-Lens\\\texttt{SAE.from\_pretrained}};
\node[greybox, below=0.6cm of saelens] (features) {Active features\\$\mathcal{F}(x)$};

% ---- AI Agent column (right) ----
\node[redbox, right=2.4cm of tlens] (beliefs) {Belief state\\$q(\vect{s}_t)$};
\node[redbox, below=0.6cm of beliefs] (efe) {\EFE\ selection\\Eq.~(\ref*{eq:efe_approx})};
\node[redbox, below=0.6cm of efe] (action) {Intervention\\$(l^*,k^*,u^*)$};
\node[redbox, below=0.6cm of action] (observe) {Observe\\$o_t$, $\delta_t$};
\node[redbox, below=0.6cm of observe] (update) {Belief update\\Eq.~(\ref*{eq:belief_update})};

% ---- Output column ----
\node[greenbox, below=0.7cm of features, xshift=3.0cm] (circuit) {Circuit $\mathcal{C}$};
\node[greenbox, below=0.5cm of circuit] (graph) {Attribution\\graph $G$};
\node[greenbox, below=0.5cm of graph] (preds) {Predictions $\mathcal{P}$};

% ---- pymdp label ----
\node[above=0.1cm of beliefs, font=\footnotesize\itshape, text=red!70!black]
    {\pymdp\ Agent};

% ---- Grouping boxes ----
\begin{scope}[on background layer]
  \node[groupbox, fit=(tlens)(cache)(saelens)(features),
        label={[font=\scriptsize, text=black!50]above:LLM + SAE pipeline}] {};
  \node[groupbox, draw=red!40, fit=(beliefs)(efe)(action)(observe)(update),
        label={[font=\scriptsize, text=red!60!black]above:Active Inference loop}] {};
  \node[groupbox, draw=green!40, fit=(circuit)(graph)(preds),
        label={[font=\scriptsize, text=green!60!black]above:Outputs}] {};
\end{scope}

% ---- Arrows ----
\draw[arr] (prompt) -- (tlens);
\draw[arr] (tlens) -- (cache);
\draw[arr] (cache) -- (saelens);
\draw[arr] (saelens) -- (features);

% features -> beliefs (initialise)
\draw[arr] (features.east) -- ++(0.4,0) |- (beliefs.west)
    node[midway, above, font=\scriptsize]{init};

% EFE loop
\draw[arr] (beliefs) -- (efe);
\draw[arr] (efe) -- (action);
\draw[arr] (action.west) -- ++(-0.5,0) |- (cache.east)
    node[pos=0.3, right, font=\scriptsize]{intervene};
\draw[arr] (cache.east) -- ++(0.5,0) |- (observe.west)
    node[pos=0.5, right, font=\scriptsize]{$o_t$};
\draw[arr] (observe) -- (update);
\draw[arr] (update.north) -- ++(0, 0.15) -| (beliefs.south)
    node[pos=0.1, right, font=\scriptsize]{$q_{t+1}$};

% Convergence check
\draw[darr] (update.east) -- ++(0.5,0)
    node[right, font=\scriptsize]{converged?};

% Output arrows
\draw[arr] (update.south) -- ++(0,-0.3) -| (circuit.north);
\draw[arr] (circuit) -- (graph);
\draw[arr] (graph) -- (preds);

\end{tikzpicture}
%
}
\caption{System architecture of \ACD{}. The \texttt{circuit-tracer}
  pipeline (left) extracts candidate features via EAP and pruning.
  The Active Inference Selector (right) iteratively scores, selects,
  and ablates features using an uncertainty-weighted strategy.}
\label{fig:architecture}
\end{figure}

\ACD{} consists of three layers:

\begin{enumerate}
\item \textbf{Attribution Graph Backend.} Anthropic's
  \texttt{circuit-tracer} library~\cite{Anthropic2025CT} generates
  attribution graphs via Edge Attribution Patching (EAP) with GemmaScope
  transcoders for Gemma-2-2B.  The graph contains active transcoder
  features, an adjacency matrix encoding feature interactions, and
  activation values.

\item \textbf{Active Inference Selector.}  An uncertainty-weighted
  scoring function selects the next feature to intervene on, combining
  graph-structural importance (exploitation) with per-feature uncertainty
  (exploration).  Per-layer priors are learned online from observed
  causal effects.

\item \textbf{Intervention Engine.}  Executes feature-level ablations
  and steering via the \texttt{feature\_intervention} API, which
  intervenes at the transcoder level with proper network propagation
  through the underlying TransformerLens~\cite{nanda2022transformerlens}
  model.
\end{enumerate}


\subsection{Candidate Feature Extraction}

Given a prompt, the attribution graph backend produces:
\begin{itemize}
\item Active features $\{(l_i, p_i, f_i)\}$ where $l$ is layer, $p$
  is token position, and $f$ is feature index.
\item Adjacency matrix $\mat{W} \in \mathbb{R}^{n \times n}$ encoding
  feature-to-feature attribution weights.
\item Activation values $\vect{a} \in \mathbb{R}^n$.
\end{itemize}

The graph is pruned to retain features with influence above a threshold
(default 80\%).  Each retained feature $i$ receives a normalised
importance score:
\begin{equation}
  \text{imp}(i) = \frac{\sum_j |W_{ij}| + \sum_j |W_{ji}|}
                       {\max_k \left(\sum_j |W_{kj}| + \sum_j |W_{jk}|\right)}
\end{equation}

Candidates are sampled across layers (up to $k$ per layer) to ensure
diversity.


\subsection{Active Inference Selector}

\begin{figure}[t]
\centering
\resizebox{\columnwidth}{!}{%
% figures/efe_pipeline.tex -- Scoring function pipeline (TikZ)
% Shows the pragmatic + epistemic scoring of the ActiveInferenceSelector

\begin{tikzpicture}[
    font=\small,
    node distance = 0.6cm and 1.2cm,
    box/.style = {
        rectangle, draw=black!60, fill=blue!6,
        rounded corners=2pt, minimum width=2.2cm, minimum height=0.6cm,
        align=center, line width=0.4pt
    },
    formula/.style = {
        rectangle, draw=red!50!black, fill=red!5,
        rounded corners=2pt, minimum width=3.0cm, minimum height=0.7cm,
        align=center, line width=0.4pt, font=\small
    },
    output/.style = {
        rectangle, draw=green!50!black, fill=green!6,
        rounded corners=2pt, minimum width=2.2cm, minimum height=0.6cm,
        align=center, line width=0.4pt
    },
    arr/.style = {-{Latex[length=2mm]}, thick, draw=black!50},
]

% Inputs
\node[box] (imp) {$\text{imp}(i)$\\graph importance};
\node[box, below=of imp] (unc) {$u(i) = 1$\\uncertainty};
\node[box, below=of unc] (lambda) {$\lambda_\ell$\\layer prior};

% Pragmatic term
\node[formula, right=1.6cm of imp] (pragmatic)
    {Pragmatic:\\$\text{imp}(i) \times \lambda_\ell$};

% Epistemic term
\node[formula, right=1.6cm of unc] (epistemic)
    {Epistemic:\\$w_{\text{explore}} \times u(i)$};

% Combine
\node[formula, right=1.6cm of epistemic, yshift=0.6cm] (combine)
    {$S(i) = $ pragmatic\\$+$ epistemic};

% Selection
\node[output, right=1.2cm of combine] (select)
    {$i^* = \arg\max_i S(i)$};

% Update box
\node[output, below=0.8cm of select] (update_box)
    {After ablation:\\$u(i^*) \gets 0$\\$\lambda_\ell \gets \lambda_\ell + |\Delta\text{KL}|$};

% Arrows
\draw[arr] (imp) -- (pragmatic);
\draw[arr] (lambda.east) -- ++(0.4,0) |- (pragmatic.south west);
\draw[arr] (unc) -- (epistemic);
\draw[arr] (pragmatic) -| (combine);
\draw[arr] (epistemic) -- (combine);
\draw[arr] (combine) -- (select);
\draw[arr] (select) -- (update_box);
\draw[arr, dashed, draw=black!30] (update_box.west) -- ++(-3.0,0) |- (unc.east)
    node[pos=0.1, below, font=\scriptsize]{next iteration};

\end{tikzpicture}
%
}
\caption{Scoring function pipeline. Each candidate receives a pragmatic
  score (graph importance $\times$ layer prior) and an epistemic bonus
  (exploration weight $\times$ uncertainty). After ablation, the
  uncertainty is zeroed and the layer prior is updated based on
  observed KL divergence.}
\label{fig:scoring}
\end{figure}

The selector scores each unobserved candidate feature $i$ as:
\begin{equation}
\label{eq:score}
  \text{score}(i) = \underbrace{\text{imp}(i) \cdot \lambda_{\ell(i)}}_{\text{pragmatic}}
  + \underbrace{u(i) \cdot \omega_e}_{\text{epistemic}}
\end{equation}

where:
\begin{itemize}
\item $\text{imp}(i) \in [0, 1]$ is the normalised graph influence.
\item $\lambda_\ell$ is a learned per-layer prior, initialised to 1.
\item $u(i) \in [0, 1]$ is the feature uncertainty, initialised to 1.
\item $\omega_e > 0$ is the exploration weight (default 2.0).
\end{itemize}

The feature with the highest score is selected:
$i^* = \arg\max_{i \notin \mathcal{O}} \text{score}(i)$.


\subsection{Belief Updating}

After ablating feature $i^*$ and observing the KL divergence
$\text{KL}_{i^*}$, the selector updates its state:

\begin{enumerate}
\item \textbf{Uncertainty reduction.}
  $u(i^*) \leftarrow 0$ (fully observed).
  For all features $j$ in the same layer:
  $u(j) \leftarrow 0.7 \cdot u(j)$ (partial transfer).
  For all features $j$ at adjacent positions:
  $u(j) \leftarrow 0.9 \cdot u(j)$.

\item \textbf{Layer prior update.}
  Let $\bar{\text{KL}}_\ell$ be the running mean KL for layer $\ell$ and
  $\bar{\text{KL}}_{\text{global}}$ be the overall mean.  Then:
  \begin{equation}
    \lambda_\ell \leftarrow 1 + 0.5 \left(
      \frac{\bar{\text{KL}}_\ell}{\bar{\text{KL}}_{\text{global}}} - 1
    \right)
  \end{equation}
  This upweights layers whose features have high causal impact and
  downweights uninformative layers.
\end{enumerate}

The pragmatic--epistemic decomposition in~\cref{eq:score} mirrors the
Expected Free Energy decomposition in active
inference~\cite{Friston2015,DaCosta2020}: the pragmatic term drives
exploitation of known high-value features, while the epistemic term
drives exploration of uncertain features.  Unlike a full POMDP agent,
the selector operates in a bandit-like setting where each feature is
intervened on at most once, making the scoring function a lightweight
but effective approximation.


\subsection{Intervention Engine}

\begin{figure}[t]
\centering
\resizebox{0.85\columnwidth}{!}{%
% figures/circuit_flow.tex -- Flow of a single intervention step (TikZ)
% Compact version for IEEE single-column

\begin{tikzpicture}[
    font=\scriptsize,
    node distance = 0.5cm,
    flowstep/.style = {
        rectangle, draw=black!60, fill=white,
        rounded corners=2pt, minimum width=2.2cm, minimum height=0.45cm,
        align=center, line width=0.4pt
    },
    decision/.style = {
        diamond, draw=blue!60!black, fill=blue!6,
        aspect=2.5, minimum width=0.8cm, align=center,
        inner sep=1pt, font=\scriptsize, line width=0.4pt
    },
    arr/.style = {-{Latex[length=1.5mm]}, semithick, draw=black!50},
    ann/.style = {font=\tiny, text=black!60},
]

\node[flowstep] (s1) {EAP + prune};
\node[flowstep, below=of s1] (s2) {Extract candidates $(l, p, f)$};
\node[flowstep, below=of s2] (s3) {Score: $S = \text{imp}\cdot\lambda_\ell + \omega_e\cdot u$};
\node[flowstep, below=of s3] (s4) {Select $i^* = \arg\max S$};
\node[flowstep, below=of s4] (s5) {\texttt{feature\_intervention}$(l,p,f,0)$};
\node[decision, below=0.6cm of s5] (d1) {$\Delta$KL?};
\node[flowstep, below right=0.6cm and 0.2cm of d1] (s6a) {$u(i^*)\!\gets\!0$, update $\lambda_\ell$};
\node[flowstep, below left=0.6cm and 0.2cm of d1] (s6b) {$u(i^*)\!\gets\!0$, no update};

\draw[arr] (s1) -- (s2);
\draw[arr] (s2) -- (s3);
\draw[arr] (s3) -- (s4);
\draw[arr] (s4) -- (s5);
\draw[arr] (s5) -- (d1);
\draw[arr] (d1) -- (s6a) node[midway, right, ann]{$>0$};
\draw[arr] (d1) -- (s6b) node[midway, left, ann]{$=0$};

\draw[arr, dashed] (s6a.north) -- ++(0, 0.15) -| ([xshift=1.0cm]s3.east) -- (s3.east)
    node[pos=0.15, right, ann]{next};
\draw[arr, dashed] (s6b.north) -- ++(0, 0.15) -| ([xshift=-1.0cm]s3.west) -- (s3.west);

\node[ann, right=0.8cm of s4] {$\le B$ steps};

\end{tikzpicture}
%
}
\caption{Flow of a single intervention step. The selector scores
  candidates, selects the best, ablates it via
  \texttt{feature\_intervention}, and updates its uncertainty and
  layer prior based on the observed KL divergence.}
\label{fig:flow}
\end{figure}

The intervention engine implements two manipulation types via the
\texttt{feature\_intervention} API:

\textbf{Ablation:} Sets transcoder feature $(l, p, f)$ to zero:
\begin{equation}
  \text{logits}_{\text{abl}} = \text{feature\_intervention}(\text{prompt},
  [(l, p, f, 0)])
\end{equation}

\textbf{Feature Steering:} Scales activation by multiplier $m$:
\begin{equation}
  \text{logits}_{\text{steer}} = \text{feature\_intervention}(\text{prompt},
  [(l, p, f, a_i \cdot m)])
\end{equation}

where $a_i$ is the clean activation value of feature $i$.

Effect is measured via KL divergence between the clean and intervened
output distributions:
\begin{equation}
  \text{KL}_i = D_{\text{KL}}\!\left(
    \text{softmax}(\text{logits}_{\text{interv}})
    \;\|\;
    \text{softmax}(\text{logits}_{\text{clean}})
  \right)
\end{equation}

The \texttt{feature\_intervention} API correctly propagates the
intervention through the replacement model's transcoder structure,
ensuring that the measured effect reflects the true causal contribution
of the feature rather than a residual-stream approximation.
