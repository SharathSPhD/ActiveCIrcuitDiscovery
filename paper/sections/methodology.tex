% ---- methodology.tex ----

This section presents the Active Circuit Discovery (\ACD) framework in full detail.
Figure~\ref{fig:architecture} provides an overview of the end-to-end pipeline.

\begin{figure}[t]
  \centering
  % figures/architecture.tex -- System architecture diagram (TikZ)
% Rendered as % figures/architecture.tex -- System architecture diagram (TikZ)
% Rendered as % figures/architecture.tex -- System architecture diagram (TikZ)
% Rendered as \input{figures/architecture} inside a figure environment

\begin{tikzpicture}[
    font=\small,
    node distance = 0.9cm and 1.5cm,
    box/.style = {
        rectangle, draw=black!70, fill=blue!8,
        rounded corners=3pt, minimum width=2.6cm, minimum height=0.7cm,
        align=center, line width=0.5pt
    },
    redbox/.style = {
        rectangle, draw=red!60!black, fill=red!8,
        rounded corners=3pt, minimum width=2.6cm, minimum height=0.7cm,
        align=center, line width=0.5pt
    },
    greenbox/.style = {
        rectangle, draw=green!50!black, fill=green!8,
        rounded corners=3pt, minimum width=2.6cm, minimum height=0.7cm,
        align=center, line width=0.5pt
    },
    greybox/.style = {
        rectangle, draw=black!40, fill=black!4,
        rounded corners=3pt, minimum width=2.6cm, minimum height=0.7cm,
        align=center, line width=0.5pt
    },
    arr/.style = {-{Latex[length=2mm]}, thick, draw=black!60},
    darr/.style = {{Latex[length=2mm]}-{Latex[length=2mm]}, thick, draw=black!40, dashed},
    groupbox/.style = {
        draw=black!30, dashed, fill=none, rounded corners=6pt, inner sep=6pt
    }
]

% ---- Input ----
\node[box] (prompt) {Prompt $x$};

% ---- LLM column ----
\node[greybox, below=0.7cm of prompt] (tlens) {TransformerLens\\\texttt{HookedTransformer}};
\node[greybox, below=0.6cm of tlens] (cache) {Residual stream\\cache $\{\vect{h}_l\}$};
\node[greybox, below=0.6cm of cache] (saelens) {SAE-Lens\\\texttt{SAE.from\_pretrained}};
\node[greybox, below=0.6cm of saelens] (features) {Active features\\$\mathcal{F}(x)$};

% ---- AI Agent column (right) ----
\node[redbox, right=2.4cm of tlens] (beliefs) {Belief state\\$q(\vect{s}_t)$};
\node[redbox, below=0.6cm of beliefs] (efe) {\EFE\ selection\\Eq.~(\ref*{eq:efe_approx})};
\node[redbox, below=0.6cm of efe] (action) {Intervention\\$(l^*,k^*,u^*)$};
\node[redbox, below=0.6cm of action] (observe) {Observe\\$o_t$, $\delta_t$};
\node[redbox, below=0.6cm of observe] (update) {Belief update\\Eq.~(\ref*{eq:belief_update})};

% ---- Output column ----
\node[greenbox, below=0.7cm of features, xshift=3.0cm] (circuit) {Circuit $\mathcal{C}$};
\node[greenbox, below=0.5cm of circuit] (graph) {Attribution\\graph $G$};
\node[greenbox, below=0.5cm of graph] (preds) {Predictions $\mathcal{P}$};

% ---- pymdp label ----
\node[above=0.1cm of beliefs, font=\footnotesize\itshape, text=red!70!black]
    {\pymdp\ Agent};

% ---- Grouping boxes ----
\begin{scope}[on background layer]
  \node[groupbox, fit=(tlens)(cache)(saelens)(features),
        label={[font=\scriptsize, text=black!50]above:LLM + SAE pipeline}] {};
  \node[groupbox, draw=red!40, fit=(beliefs)(efe)(action)(observe)(update),
        label={[font=\scriptsize, text=red!60!black]above:Active Inference loop}] {};
  \node[groupbox, draw=green!40, fit=(circuit)(graph)(preds),
        label={[font=\scriptsize, text=green!60!black]above:Outputs}] {};
\end{scope}

% ---- Arrows ----
\draw[arr] (prompt) -- (tlens);
\draw[arr] (tlens) -- (cache);
\draw[arr] (cache) -- (saelens);
\draw[arr] (saelens) -- (features);

% features -> beliefs (initialise)
\draw[arr] (features.east) -- ++(0.4,0) |- (beliefs.west)
    node[midway, above, font=\scriptsize]{init};

% EFE loop
\draw[arr] (beliefs) -- (efe);
\draw[arr] (efe) -- (action);
\draw[arr] (action.west) -- ++(-0.5,0) |- (cache.east)
    node[pos=0.3, right, font=\scriptsize]{intervene};
\draw[arr] (cache.east) -- ++(0.5,0) |- (observe.west)
    node[pos=0.5, right, font=\scriptsize]{$o_t$};
\draw[arr] (observe) -- (update);
\draw[arr] (update.north) -- ++(0, 0.15) -| (beliefs.south)
    node[pos=0.1, right, font=\scriptsize]{$q_{t+1}$};

% Convergence check
\draw[darr] (update.east) -- ++(0.5,0)
    node[right, font=\scriptsize]{converged?};

% Output arrows
\draw[arr] (update.south) -- ++(0,-0.3) -| (circuit.north);
\draw[arr] (circuit) -- (graph);
\draw[arr] (graph) -- (preds);

\end{tikzpicture}
 inside a figure environment

\begin{tikzpicture}[
    font=\small,
    node distance = 0.9cm and 1.5cm,
    box/.style = {
        rectangle, draw=black!70, fill=blue!8,
        rounded corners=3pt, minimum width=2.6cm, minimum height=0.7cm,
        align=center, line width=0.5pt
    },
    redbox/.style = {
        rectangle, draw=red!60!black, fill=red!8,
        rounded corners=3pt, minimum width=2.6cm, minimum height=0.7cm,
        align=center, line width=0.5pt
    },
    greenbox/.style = {
        rectangle, draw=green!50!black, fill=green!8,
        rounded corners=3pt, minimum width=2.6cm, minimum height=0.7cm,
        align=center, line width=0.5pt
    },
    greybox/.style = {
        rectangle, draw=black!40, fill=black!4,
        rounded corners=3pt, minimum width=2.6cm, minimum height=0.7cm,
        align=center, line width=0.5pt
    },
    arr/.style = {-{Latex[length=2mm]}, thick, draw=black!60},
    darr/.style = {{Latex[length=2mm]}-{Latex[length=2mm]}, thick, draw=black!40, dashed},
    groupbox/.style = {
        draw=black!30, dashed, fill=none, rounded corners=6pt, inner sep=6pt
    }
]

% ---- Input ----
\node[box] (prompt) {Prompt $x$};

% ---- LLM column ----
\node[greybox, below=0.7cm of prompt] (tlens) {TransformerLens\\\texttt{HookedTransformer}};
\node[greybox, below=0.6cm of tlens] (cache) {Residual stream\\cache $\{\vect{h}_l\}$};
\node[greybox, below=0.6cm of cache] (saelens) {SAE-Lens\\\texttt{SAE.from\_pretrained}};
\node[greybox, below=0.6cm of saelens] (features) {Active features\\$\mathcal{F}(x)$};

% ---- AI Agent column (right) ----
\node[redbox, right=2.4cm of tlens] (beliefs) {Belief state\\$q(\vect{s}_t)$};
\node[redbox, below=0.6cm of beliefs] (efe) {\EFE\ selection\\Eq.~(\ref*{eq:efe_approx})};
\node[redbox, below=0.6cm of efe] (action) {Intervention\\$(l^*,k^*,u^*)$};
\node[redbox, below=0.6cm of action] (observe) {Observe\\$o_t$, $\delta_t$};
\node[redbox, below=0.6cm of observe] (update) {Belief update\\Eq.~(\ref*{eq:belief_update})};

% ---- Output column ----
\node[greenbox, below=0.7cm of features, xshift=3.0cm] (circuit) {Circuit $\mathcal{C}$};
\node[greenbox, below=0.5cm of circuit] (graph) {Attribution\\graph $G$};
\node[greenbox, below=0.5cm of graph] (preds) {Predictions $\mathcal{P}$};

% ---- pymdp label ----
\node[above=0.1cm of beliefs, font=\footnotesize\itshape, text=red!70!black]
    {\pymdp\ Agent};

% ---- Grouping boxes ----
\begin{scope}[on background layer]
  \node[groupbox, fit=(tlens)(cache)(saelens)(features),
        label={[font=\scriptsize, text=black!50]above:LLM + SAE pipeline}] {};
  \node[groupbox, draw=red!40, fit=(beliefs)(efe)(action)(observe)(update),
        label={[font=\scriptsize, text=red!60!black]above:Active Inference loop}] {};
  \node[groupbox, draw=green!40, fit=(circuit)(graph)(preds),
        label={[font=\scriptsize, text=green!60!black]above:Outputs}] {};
\end{scope}

% ---- Arrows ----
\draw[arr] (prompt) -- (tlens);
\draw[arr] (tlens) -- (cache);
\draw[arr] (cache) -- (saelens);
\draw[arr] (saelens) -- (features);

% features -> beliefs (initialise)
\draw[arr] (features.east) -- ++(0.4,0) |- (beliefs.west)
    node[midway, above, font=\scriptsize]{init};

% EFE loop
\draw[arr] (beliefs) -- (efe);
\draw[arr] (efe) -- (action);
\draw[arr] (action.west) -- ++(-0.5,0) |- (cache.east)
    node[pos=0.3, right, font=\scriptsize]{intervene};
\draw[arr] (cache.east) -- ++(0.5,0) |- (observe.west)
    node[pos=0.5, right, font=\scriptsize]{$o_t$};
\draw[arr] (observe) -- (update);
\draw[arr] (update.north) -- ++(0, 0.15) -| (beliefs.south)
    node[pos=0.1, right, font=\scriptsize]{$q_{t+1}$};

% Convergence check
\draw[darr] (update.east) -- ++(0.5,0)
    node[right, font=\scriptsize]{converged?};

% Output arrows
\draw[arr] (update.south) -- ++(0,-0.3) -| (circuit.north);
\draw[arr] (circuit) -- (graph);
\draw[arr] (graph) -- (preds);

\end{tikzpicture}
 inside a figure environment

\begin{tikzpicture}[
    font=\small,
    node distance = 0.9cm and 1.5cm,
    box/.style = {
        rectangle, draw=black!70, fill=blue!8,
        rounded corners=3pt, minimum width=2.6cm, minimum height=0.7cm,
        align=center, line width=0.5pt
    },
    redbox/.style = {
        rectangle, draw=red!60!black, fill=red!8,
        rounded corners=3pt, minimum width=2.6cm, minimum height=0.7cm,
        align=center, line width=0.5pt
    },
    greenbox/.style = {
        rectangle, draw=green!50!black, fill=green!8,
        rounded corners=3pt, minimum width=2.6cm, minimum height=0.7cm,
        align=center, line width=0.5pt
    },
    greybox/.style = {
        rectangle, draw=black!40, fill=black!4,
        rounded corners=3pt, minimum width=2.6cm, minimum height=0.7cm,
        align=center, line width=0.5pt
    },
    arr/.style = {-{Latex[length=2mm]}, thick, draw=black!60},
    darr/.style = {{Latex[length=2mm]}-{Latex[length=2mm]}, thick, draw=black!40, dashed},
    groupbox/.style = {
        draw=black!30, dashed, fill=none, rounded corners=6pt, inner sep=6pt
    }
]

% ---- Input ----
\node[box] (prompt) {Prompt $x$};

% ---- LLM column ----
\node[greybox, below=0.7cm of prompt] (tlens) {TransformerLens\\\texttt{HookedTransformer}};
\node[greybox, below=0.6cm of tlens] (cache) {Residual stream\\cache $\{\vect{h}_l\}$};
\node[greybox, below=0.6cm of cache] (saelens) {SAE-Lens\\\texttt{SAE.from\_pretrained}};
\node[greybox, below=0.6cm of saelens] (features) {Active features\\$\mathcal{F}(x)$};

% ---- AI Agent column (right) ----
\node[redbox, right=2.4cm of tlens] (beliefs) {Belief state\\$q(\vect{s}_t)$};
\node[redbox, below=0.6cm of beliefs] (efe) {\EFE\ selection\\Eq.~(\ref*{eq:efe_approx})};
\node[redbox, below=0.6cm of efe] (action) {Intervention\\$(l^*,k^*,u^*)$};
\node[redbox, below=0.6cm of action] (observe) {Observe\\$o_t$, $\delta_t$};
\node[redbox, below=0.6cm of observe] (update) {Belief update\\Eq.~(\ref*{eq:belief_update})};

% ---- Output column ----
\node[greenbox, below=0.7cm of features, xshift=3.0cm] (circuit) {Circuit $\mathcal{C}$};
\node[greenbox, below=0.5cm of circuit] (graph) {Attribution\\graph $G$};
\node[greenbox, below=0.5cm of graph] (preds) {Predictions $\mathcal{P}$};

% ---- pymdp label ----
\node[above=0.1cm of beliefs, font=\footnotesize\itshape, text=red!70!black]
    {\pymdp\ Agent};

% ---- Grouping boxes ----
\begin{scope}[on background layer]
  \node[groupbox, fit=(tlens)(cache)(saelens)(features),
        label={[font=\scriptsize, text=black!50]above:LLM + SAE pipeline}] {};
  \node[groupbox, draw=red!40, fit=(beliefs)(efe)(action)(observe)(update),
        label={[font=\scriptsize, text=red!60!black]above:Active Inference loop}] {};
  \node[groupbox, draw=green!40, fit=(circuit)(graph)(preds),
        label={[font=\scriptsize, text=green!60!black]above:Outputs}] {};
\end{scope}

% ---- Arrows ----
\draw[arr] (prompt) -- (tlens);
\draw[arr] (tlens) -- (cache);
\draw[arr] (cache) -- (saelens);
\draw[arr] (saelens) -- (features);

% features -> beliefs (initialise)
\draw[arr] (features.east) -- ++(0.4,0) |- (beliefs.west)
    node[midway, above, font=\scriptsize]{init};

% EFE loop
\draw[arr] (beliefs) -- (efe);
\draw[arr] (efe) -- (action);
\draw[arr] (action.west) -- ++(-0.5,0) |- (cache.east)
    node[pos=0.3, right, font=\scriptsize]{intervene};
\draw[arr] (cache.east) -- ++(0.5,0) |- (observe.west)
    node[pos=0.5, right, font=\scriptsize]{$o_t$};
\draw[arr] (observe) -- (update);
\draw[arr] (update.north) -- ++(0, 0.15) -| (beliefs.south)
    node[pos=0.1, right, font=\scriptsize]{$q_{t+1}$};

% Convergence check
\draw[darr] (update.east) -- ++(0.5,0)
    node[right, font=\scriptsize]{converged?};

% Output arrows
\draw[arr] (update.south) -- ++(0,-0.3) -| (circuit.north);
\draw[arr] (circuit) -- (graph);
\draw[arr] (graph) -- (preds);

\end{tikzpicture}

  \caption{Overview of the \ACD\ framework. The outer loop applies the
           Active Inference agent; the inner loop performs interventions
           via TransformerLens and SAE-Lens. Belief state $q(\vect{s})$ is
           updated after each intervention observation.}
  \label{fig:architecture}
\end{figure}

\subsection{Formal Problem Statement}

Let $\mathcal{M}$ be a transformer LLM with $L$ layers. Given a prompt $x$ and a
target behaviour $b$ (e.g.\ the probability mass assigned to a correct completion
token), the circuit discovery problem is to find the minimal set
$\mathcal{C} \subseteq \mathcal{N}$ of model components (nodes in the computational
graph $\mathcal{N}$) such that ablating all components outside $\mathcal{C}$ does not
significantly reduce $b$~\cite{Wang2022,Conmy2023}.

In the SAE feature representation, each node corresponds to a latent direction
$\vect{f}_{l,k} \in \mathbb{R}^d$ in the residual stream at layer $l$, where $k$
indexes over the $K_l$ features learned by the SAE at that layer. The feature is
``active'' on prompt $x$ if its SAE encoder activation exceeds a threshold
$\theta > 0$:
\begin{equation}
  a_{l,k}(x) = \operatorname{ReLU}\!\bigl(\mat{W}_e^{(l)} \vect{h}_l(x) + \vect{b}_e^{(l)}\bigr)_k
  \;\geq\; \theta,
  \label{eq:sae_activation}
\end{equation}
where $\vect{h}_l(x)$ is the residual stream at layer $l$ and position
corresponding to the last token of $x$, and $\mat{W}_e^{(l)}$, $\vect{b}_e^{(l)}$
are the SAE encoder weights and bias.

The set of active features across all layers is denoted
$\mathcal{F}(x) = \{(l,k) : a_{l,k}(x) \geq \theta\}$. The goal is to identify
the subset $\mathcal{C}(x) \subseteq \mathcal{F}(x)$ that is causally sufficient for
behaviour $b$.

\subsection{Generative Model and State Space}
\label{sec:genmodel}

The Active Inference agent maintains a generative model
$p(\vect{o}, \vect{s}) = p(\vect{o} \mid \vect{s})\, p(\vect{s})$ over:
\begin{itemize}
  \item \textit{Hidden states} $\vect{s}$: a categorical distribution over the
        $N$ active features, representing beliefs about feature importance for
        the circuit. The state space has dimension $N + 1$ (including an
        ``irrelevant feature'' state).
  \item \textit{Observations} $\vect{o}$: the discretised effect size of an
        intervention, mapped to one of three categories: \{low, medium, high\},
        corresponding to effect sizes below 0.3, between 0.3 and 0.7, and
        above 0.7 standard deviations of the logit difference, respectively.
  \item \textit{Actions} $\vect{u}$: the choice of intervention type from
        \{\texttt{ablation}, \texttt{activation\_patching}, \texttt{mean\_ablation}\}.
\end{itemize}

The generative model parameters are:
\begin{align}
  \mat{A} &\in \mathbb{R}^{3 \times (N+1)}, \quad \mat{A}_{ok} = p(o \mid s = k), \\
  \mat{B} &\in \mathbb{R}^{(N+1) \times (N+1) \times 3}, \quad \mat{B}_{k'ku} = p(s' = k' \mid s = k, u), \\
  \vect{c} &\in \mathbb{R}^{3}, \quad c_o = \log p(o), \\
  \vect{d} &\in \mathbb{R}^{N+1}, \quad d_k = p(s_0 = k),
\end{align}
where $\mat{A}$ encodes how feature importance determines intervention effects,
$\mat{B}$ encodes how beliefs transition as interventions eliminate candidates,
$\vect{c}$ encodes the preference for high-effect (informative) observations, and
$\vect{d}$ encodes the uniform prior over initial feature importance.

The observation model $\mat{A}$ is initialised as follows: features with high initial
activation (above the 75th percentile of $\{a_{l,k}\}$) are assigned a higher
probability of producing high-effect observations; features with low initial activation
are assigned a higher probability of low-effect observations. This encoding captures
the prior expectation that highly active features are more likely to be circuit
members.

\subsection{Expected Free Energy Computation}
\label{sec:efe}

Given the current belief $q(\vect{s}_t) = \operatorname{Cat}(\vect{\pi}_t)$ and a
candidate feature $(l, k)$ to intervene on under action $u$, the Expected Free Energy
is computed as:
\begin{align}
  \EFE(l,k,u) &= \underbrace{-\sum_o \hat{p}(o) \log \hat{p}(o)}_{\text{epistemic}}
                + \underbrace{\vect{c}^\top \hat{\vect{p}}(o)}_{\text{pragmatic}}
  \label{eq:efe_approx}
\end{align}
where $\hat{p}(o) = \sum_k \mat{A}_{ok}\, \pi_{t,k}$ is the predictive distribution
over observations under the current beliefs. The epistemic term is the Shannon entropy
of the predictive observation distribution: interventions on features for which the
agent is most uncertain about the outcome contribute most to epistemic value. The
pragmatic term reflects the agent's preference for high-effect observations.

Figure~\ref{fig:efe_pipeline} illustrates this computation. In the \pymdp\
implementation, the full \EFE\ is computed via:
\begin{equation}
  \EFE(\pi) = \mathbb{H}[p(\vect{o}_\tau \mid \pi)] -
              \E[\mathbb{H}[p(\vect{o}_\tau \mid \vect{s}_\tau)]] +
              \E[\vect{c}^\top \log \hat{\vect{p}}(\vect{o}_\tau \mid \pi)],
  \label{eq:efe_full}
\end{equation}
where the first two terms together form the mutual information between states and
observations (epistemic value), and the third term is the expected log-preference
(pragmatic value)~\cite{DaCosta2020,Friston2015}.

\begin{figure}[t]
  \centering
  % figures/efe_pipeline.tex -- Scoring function pipeline (TikZ)
% Shows the pragmatic + epistemic scoring of the ActiveInferenceSelector

\begin{tikzpicture}[
    font=\small,
    node distance = 0.6cm and 1.2cm,
    box/.style = {
        rectangle, draw=black!60, fill=blue!6,
        rounded corners=2pt, minimum width=2.2cm, minimum height=0.6cm,
        align=center, line width=0.4pt
    },
    formula/.style = {
        rectangle, draw=red!50!black, fill=red!5,
        rounded corners=2pt, minimum width=3.0cm, minimum height=0.7cm,
        align=center, line width=0.4pt, font=\small
    },
    output/.style = {
        rectangle, draw=green!50!black, fill=green!6,
        rounded corners=2pt, minimum width=2.2cm, minimum height=0.6cm,
        align=center, line width=0.4pt
    },
    arr/.style = {-{Latex[length=2mm]}, thick, draw=black!50},
]

% Inputs
\node[box] (imp) {$\text{imp}(i)$\\graph importance};
\node[box, below=of imp] (unc) {$u(i) = 1$\\uncertainty};
\node[box, below=of unc] (lambda) {$\lambda_\ell$\\layer prior};

% Pragmatic term
\node[formula, right=1.6cm of imp] (pragmatic)
    {Pragmatic:\\$\text{imp}(i) \times \lambda_\ell$};

% Epistemic term
\node[formula, right=1.6cm of unc] (epistemic)
    {Epistemic:\\$w_{\text{explore}} \times u(i)$};

% Combine
\node[formula, right=1.6cm of epistemic, yshift=0.6cm] (combine)
    {$S(i) = $ pragmatic\\$+$ epistemic};

% Selection
\node[output, right=1.2cm of combine] (select)
    {$i^* = \arg\max_i S(i)$};

% Update box
\node[output, below=0.8cm of select] (update_box)
    {After ablation:\\$u(i^*) \gets 0$\\$\lambda_\ell \gets \lambda_\ell + |\Delta\text{KL}|$};

% Arrows
\draw[arr] (imp) -- (pragmatic);
\draw[arr] (lambda.east) -- ++(0.4,0) |- (pragmatic.south west);
\draw[arr] (unc) -- (epistemic);
\draw[arr] (pragmatic) -| (combine);
\draw[arr] (epistemic) -- (combine);
\draw[arr] (combine) -- (select);
\draw[arr] (select) -- (update_box);
\draw[arr, dashed, draw=black!30] (update_box.west) -- ++(-3.0,0) |- (unc.east)
    node[pos=0.1, below, font=\scriptsize]{next iteration};

\end{tikzpicture}

  \caption{Computation graph for Expected Free Energy ($\EFE$). Shaded nodes are
           computed quantities; white nodes are generative model parameters.
           The agent selects the feature-intervention pair that minimises $-\EFE$.}
  \label{fig:efe_pipeline}
\end{figure}

\subsection{Belief Updating and Correspondence Measurement}
\label{sec:belief_update}

After executing intervention $(l,k,u)$ and observing effect category $o_t$, the agent
updates its belief state via Bayesian filtering (variational message passing in
\pymdp):
\begin{align}
  q(\vect{s}_{t+1}) &= \softmax\!\left(\log \vect{d} + \sum_{s}\log\mat{A}_{\cdot,s}\, q(s_t) + \log \mat{B}_{\cdot,s_t,u_t}\right).
  \label{eq:belief_update}
\end{align}
Observation model parameters are updated online via a Dirichlet prior
$\mat{A}^{(a)}$ (concentration parameter $a_0 = 1$):
\begin{equation}
  \mat{A}^{(a)}_{o,k} \mathrel{+}= q(s_t = k) \cdot \mathbf{1}[o_t = o],
  \label{eq:param_learning}
\end{equation}
allowing the agent to refine its expectation of how feature importance maps to
intervention effects as the experiment progresses.

\textbf{Correspondence metric.} The primary correspondence measure (RQ1) is the
Spearman rank correlation between the vector of feature importance scores derived from
the agent's belief state $\{d_k = \pi_{T,k}\}_{k=1}^N$ and the vector of empirical
direct effect magnitudes $\{e_k = |\Delta \text{logit}(k)|\}_{k=1}^N$ obtained by
ablating each feature once and measuring the change in the target token logit. Unlike
Pearson correlation, Spearman correlation is robust to monotone nonlinearity and does
not require the two quantities to be on the same scale. A threshold of
$\rho_s > 0.4$, $p < 0.05$ is required to declare correspondence for RQ1.

\subsection{Intervention Protocol}
\label{sec:intervention}

Three intervention types are implemented, following the causal scrubbing
taxonomy~\cite{Chan2022}:

\textbf{Zero ablation.} The SAE feature activation at position $(l,k)$ is set to
zero by subtracting the feature's contribution from the residual stream:
$\tilde{\vect{h}}_l = \vect{h}_l - a_{l,k}\, \vect{f}_{l,k}$,
where $\vect{f}_{l,k}$ is the SAE decoder direction. The model is then rerun
from layer $l$ onwards with the modified residual stream, and the change in the target
logit is recorded.

\textbf{Mean ablation.} The feature activation is replaced by its mean over a
100-sentence reference corpus $\mathcal{D}_\text{ref}$ drawn from the Pile:
$\tilde{a}_{l,k} = \mathbb{E}_{x \sim \mathcal{D}_\text{ref}}[a_{l,k}(x)]$.
This preserves the typical contribution of the feature while removing its
input-specific component, providing a more conservative causal estimate than zero
ablation~\cite{Chan2022,Conmy2023}.

\textbf{Activation patching.} The residual stream at layer $l$ from a clean forward
pass on a corrupted input $\tilde{x}$ (constructed by replacing named geographic
entities) is used in place of the clean stream. Formally:
$\vect{h}_l^{\text{patch}} = \vect{h}_l(\tilde{x}) + (\vect{h}_l(x) - \vect{h}_l(\tilde{x})) \cdot a_{l,k}(x) / \norm{\vect{h}_l(x)}$.
This enables attribution of output differences to specific layers~\cite{Wang2022}.

\subsection{Attribution Graph Construction}
\label{sec:attribution_graph}

After completing the Active Inference loop, \ACD\ constructs an attribution graph
$G = (V, E)$ where each node $v \in V$ corresponds to an active SAE feature and each
directed edge $(v_i, v_j) \in E$ is weighted by the partial correlation between the
ablation effect of feature $i$ and the change in feature $j$'s activation. This
follows the edge attribution paradigm of~\cite{Syed2023} adapted to the SAE feature
basis. The graph is rendered using the \texttt{circuit-tracer}~\cite{Anthropic2025CT}
and CircuitsVis~\cite{Conmy2024} libraries, which produce interactive HTML
visualisations of node importances and edge weights.

\subsection{Novel Prediction Generation}
\label{sec:predictions}

A key capability of the Active Inference framing is its ability to generate testable
predictions from the learned generative model. Three prediction classes are
instantiated:

\textbf{Precision-weighted attention.} The agent's precision estimate
$\gamma = 1/\bar{\sigma}^2$, where $\bar{\sigma}^2$ is the mean posterior uncertainty
over feature importances, is predicted to correlate positively with the mean attention
entropy across heads in the identified circuit layer. Higher precision corresponds to
lower uncertainty, which the Active Inference framework links to sharper (lower
entropy) attention patterns~\cite{Friston2017}.

\textbf{Feature interaction transitivity.} If the belief state assigns high
connection probability $p_{\text{conn}}(i,j) > 0.5$ and $p_{\text{conn}}(j,k) > 0.5$,
the framework predicts that ablating feature $i$ will indirectly reduce the activation
of feature $k$ by at least $0.2\,\sigma$ of its baseline distribution. This is a
directional prediction derivable from the $\mat{B}$ matrix structure.

\textbf{Layer-specificity of circuit membership.} The EFE scores across layers are
predicted to follow a layer-dependent profile consistent with the residual stream
decomposition: early layers ($l < L/3$) should contribute to token-level syntax
features while middle and late layers ($l \geq L/3$) carry semantic content relevant
to factual completion~\cite{Meng2022,Elhage2021}.

Each prediction is formally stated with a testable hypothesis, a measurement
procedure, a significance threshold, and an expected effect size, enabling empirical
validation as described in Section~\ref{sec:setup}.

\subsection{Algorithm}
\label{sec:algorithm}

Algorithm~\ref{alg:acd} presents the complete \ACD\ procedure.

\begin{algorithm}[t]
  \caption{Active Circuit Discovery (\ACD)}
  \label{alg:acd}
  \begin{algorithmic}[1]
    \Input Prompt $x$, LLM $\mathcal{M}$, SAE set $\{\mathcal{S}_l\}_{l=1}^L$,
           threshold $\theta$, budget $T$
    \Output Circuit $\mathcal{C}$, belief state $q_T(\vect{s})$,
           attribution graph $G$, predictions $\mathcal{P}$
    \State $\mathcal{F}(x) \gets \{(l,k): a_{l,k}(x) \geq \theta\}$
           \Comment{Find active SAE features (Eq.~\ref{eq:sae_activation})}
    \State Initialise $\mat{A}, \mat{B}, \vect{c}, \vect{d}$ from $\mathcal{F}(x)$
    \State $q_0(\vect{s}) \gets \operatorname{Cat}(\vect{d} / \sum \vect{d})$
           \Comment{Uniform prior over features}
    \For{$t = 1, \ldots, T$}
      \For{each feature $(l,k)$ and action $u$}
        \State $\EFE(l,k,u) \gets$ compute via Eq.~\ref{eq:efe_approx}
      \EndFor
      \State $(l^*, k^*, u^*) \gets \arg\min_{l,k,u} -\EFE(l,k,u)$
      \State $o_t, \delta_t \gets$ \Call{Intervene}{$\mathcal{M}, x, l^*, k^*, u^*$}
             \Comment{Real forward pass}
      \State $q_t(\vect{s}) \gets$ update via Eq.~\ref{eq:belief_update}
      \State Update $\mat{A}^{(a)}$ via Eq.~\ref{eq:param_learning}
      \If{\Call{Converged}{$q_{t-1}, q_t$}}
            \Comment{Symmetric KL $< \delta$}
        \State \textbf{break}
      \EndIf
    \EndFor
    \State $\mathcal{C} \gets \{(l,k): \pi_{T,k} > 0.5\}$
           \Comment{High posterior importance}
    \State $G \gets$ \Call{BuildAttributionGraph}{$\mathcal{C}, x$}
    \State $\mathcal{P} \gets$ \Call{GeneratePredictions}{$q_T, G$}
    \State \Return $\mathcal{C}, q_T, G, \mathcal{P}$
  \end{algorithmic}
\end{algorithm}

Convergence is declared when the symmetric KL divergence between successive belief
distributions falls below $\delta = 0.05$:
\begin{equation}
  D_{\text{JS}}\!\left(q_{t-1}\,\|\,q_t\right)
  = \tfrac{1}{2}\KL{q_{t-1}}{q_t} + \tfrac{1}{2}\KL{q_t}{q_{t-1}} < \delta.
  \label{eq:convergence}
\end{equation}
This replaces the previous pairwise feature comparison used in early versions of the
codebase, which computed differences between the importances of two distinct features
rather than tracking the evolution of the belief distribution itself.
