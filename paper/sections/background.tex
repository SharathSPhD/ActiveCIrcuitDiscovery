% ---- background.tex ----

\subsection{Mechanistic Interpretability and Circuit Analysis}

Mechanistic interpretability traces causal pathways through neural networks by
applying targeted interventions~-- zeroing activations, swapping activations between
forward passes on different inputs, or replacing specific components with learned
approximations~-- and measuring the resulting change in model output~\cite{Pearl2009,
Geiger2021}. The transformer architecture~\cite{Vaswani2017} is especially amenable to
this analysis because its residual stream structure allows the contribution of each
component to the final logits to be decomposed additively~\cite{Elhage2021}.

Olah et al.\ introduced the concept of circuits in convolutional vision
networks~\cite{Olah2020,Cammarata2020}. Elhage et al.\ formalised the residual stream
decomposition for transformers and demonstrated in-context learning heads, induction
heads, and name-mover heads in small two-layer models~\cite{Elhage2021}. Wang et al.\
applied this framework at scale in a celebrated study of the fourteen-head IOI circuit
in GPT-2 Small, using path patching to confirm causal roles~\cite{Wang2022}. Subsequent
work characterised circuits for docstring completion~\cite{Heimersheim2023},
numerical reasoning~\cite{Hanna2023}, and Chinchilla at 70~B
parameters~\cite{Lieberum2023}.

A recurring limitation of manual circuit analysis is labour intensity. Conmy et al.\
addressed this with Automated Circuit Discovery (\ACDC), a greedy edge-pruning
algorithm that recovers circuits matching manual analyses on the IOI and docstring
tasks~\cite{Conmy2023}. Syed et al.\ proposed Edge Attribution Patching (\EAP), which
approximates intervention effects with gradient-based attribution and achieves
competitive results at a fraction of the runtime~\cite{Syed2023}. Geiger et al.\
introduced causal abstraction as a formal verification criterion for circuit
hypotheses~\cite{Geiger2021}, and Chan et al.\ developed causal scrubbing for
the same purpose~\cite{Chan2022}.

Meng et al.\ complemented circuit analysis with causal tracing applied to factual
associations in GPT models, identifying mid-layer MLP blocks as the primary locus of
stored world knowledge~\cite{Meng2022}. Lindsey et al.\ recently combined attribution
graphs constructed via \texttt{circuit-tracer}~\cite{Anthropic2025CT} with large-scale
SAE analysis to map the biology of Claude~3 Sonnet at a component
level~\cite{Lindsey2025}.

\subsection{Sparse Autoencoders for Feature Extraction}

Polysemanticity~-- the tendency of individual neurons to respond to multiple unrelated
concepts~-- has been identified as a core obstacle to mechanistic
interpretability~\cite{Olah2020,Bricken2023}. Sparse Autoencoders (SAEs) address this
by learning an overcomplete, sparse linear basis for the residual stream at each layer.
Cunningham et al.\ demonstrated that features learned by SAEs are substantially more
monosemantic than individual neurons~\cite{Cunningham2023}. Bricken et al.\ scaled
this approach to a one-layer MLP with 4,096-dimensional activations, recovering over
512 interpretable features including curve detectors and orientation-selective
units~\cite{Bricken2023}. Templeton et al.\ extended the analysis to SAE features in
Claude models, recovering structured emotional and semantic concepts~\cite{Templeton2024}.

SAE-Lens~\cite{Bloom2024} provides pre-trained SAE weights and a standardised API for
loading, querying, and applying SAEs to arbitrary residual stream activations. For
GPT-2 Small, the \texttt{gpt2-small-res-jb} release provides residual-stream SAEs at
every layer, trained on the Pile corpus. He et al.\ showed that using SAE features as
the unit of circuit analysis significantly improves circuit faithfulness compared to
attention-head-level analysis~\cite{He2024}.

\subsection{Active Inference and Expected Free Energy}

Active Inference (AI) is a unified computational framework derived from the Free
Energy Principle~\cite{Friston2010}, which posits that biological agents minimise the
surprise (negative model log-evidence) associated with their sensory states. Friston
et al.\ formalised this as variational inference: agents approximate the true posterior
over hidden states $\vect{s}$ given observations $\vect{o}$ by maintaining a
recognition distribution $q(\vect{s})$ and minimising the variational free energy
$\FE = \KL{q(\vect{s})}{p(\vect{s}\mid\vect{o})} \geq
-\log p(\vect{o})$~\cite{Friston2017}.

For planning and action selection, Da Costa et al.\ extended this to the Expected Free
Energy, defined for a policy $\pi$ over future time steps $\tau > t$
as~\cite{DaCosta2020}:
\begin{equation}
  \EFE(\pi) = \sum_{\tau > t}
    \underbrace{\E[\log p(\vect{o}_\tau) - \log q(\vect{o}_\tau \mid \pi)]}_{\text{pragmatic value}}
    - \underbrace{\E[\log q(\vect{s}_\tau \mid \pi) - \log q(\vect{s}_\tau \mid \vect{o}_\tau,\pi)]}_{\text{epistemic value}}.
  \label{eq:efe}
\end{equation}
The pragmatic value measures expected reward (preference satisfaction), while the
epistemic value measures expected information gain (uncertainty reduction). Policies are
selected according to a Boltzmann distribution over negative \EFE\ values.

Parr et al.\ provide a comprehensive treatment of Active Inference as a model of
cognition~\cite{Parr2022}. Heins et al.\ implement discrete-state Active Inference
in the \pymdp\ Python library~\cite{Pymdp2022}, which provides matrix-based
generative models $(\mat{A}, \mat{B}, \mat{C}, \mat{D})$ representing the observation
model, transition model, log-preference vector, and prior over initial states
respectively. Tschantz et al.\ demonstrated that \EFE\ minimisation recovers
reinforcement learning in the limit of pure pragmatic value~\cite{Tschantz2020}.

Sun et al.\ explored the conceptual alignment between Active Inference and neural network
interpretability, arguing that attention mechanisms implement a form of precision-weighted
prediction error minimisation~\cite{Sun2024}. The present work operationalises this
connection by embedding it within an automated circuit discovery procedure.

\subsection{Gap Analysis: Why Active Inference for Circuit Discovery}

Table~\ref{tab:method_comparison} summarises the properties of existing automated
circuit discovery methods and the proposed \ACD\ framework. Existing methods treat
intervention selection as either exhaustive~\cite{Conmy2023} or gradient-ranked and
therefore non-adaptive~\cite{Syed2023}. Neither approach explicitly models uncertainty
about circuit structure or uses that uncertainty to guide further investigation.
Active Inference addresses this gap by maintaining a full posterior over candidate
feature importances and selecting the intervention that, in expectation, most reduces
this uncertainty~-- analogous to Bayesian experimental design~\cite{MacKay2003}.

\begin{table}[t]
  \centering
  \caption{Comparison of automated circuit discovery methods.}
  \label{tab:method_comparison}
  \begin{tabularx}{\columnwidth}{lXXXX}
    \toprule
    Method & Uncertainty & Adaptive & Online & Testable \\
           & Modelling & Selection & Learning & Predictions \\
    \midrule
    \ACDC~\cite{Conmy2023}   & No  & No  & No  & No \\
    \EAP~\cite{Syed2023}     & No  & No  & No  & No \\
    Grad. Ranking            & No  & No  & No  & No \\
    \textbf{\ACD (ours)}     & Yes & Yes & Yes & Yes \\
    \bottomrule
  \end{tabularx}
\end{table}
